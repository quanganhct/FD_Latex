\section{Sort a radom array}

\begin{itemize}
\item Specification: SortArrayAscendingMachine
\item Implementation: SortArrayAscendingMachineImplementation
\end{itemize}
The algorithm is still the same as the previous machine, since I used the bubble sort algorithm. 
But we have 2 more conditions to satisfy, since the array is totally random: 
\begin{itemize}
\item $dom(new\_array) = dom(array) \land ran(new\_array) = ran(array)$
\item $\forall y \cdot y \in ran(array) \Rightarrow card(\{u | u \in dom(array) \land u \mapsto y \in array\}) = card(\{x | x \in dom(new\_array) \land x \mapsto y \in new_array\})$
\end{itemize}

That make the appearance of the POs below, which I had to prove interactively :
\begin{itemize}
\item INITIALISATION/inv14/INV
\item LOOP\_SWAP/inv14/INV
\item LOOP\_SWAP/inv13/INV
\end{itemize}

To prove the POs, I need to use the following theorem: 
\begin{itemize}
\item thm5 : $\forall g,n \cdot n \in \mathbb{N} \land g \in 1..n \to \mathbb{Z} \Rightarrow g=\{i \cdot i \in 1..n | i \mapsto g(i)\}$
\item thm7 : $\forall g,n \cdot n \in \mathbb{N} \land g \pfun 1..n \to \mathbb{Z} \Rightarrow g= \{i \cdot i \in 1..n | i \mapsto g(i)\}$
\item thm8 : $\forall A,B,C \cdot A \subseteq \mathbb{N} \land B \subseteq \mathbb{N} \land C \subseteq \mathbb{N} \land finite(A) \land finite(B) \land finite(C) \land A \cap B = \emptyset \land A \cap C = \emptyset \land B \cap C = \emptyset \Rightarrow card(A \cup B \cup C) = card(A) + card(B) + card(C)$

\item thm9 : $\forall A,B \cdot A \subseteq \mathbb{N} \land B \subseteq \mathbb{N} \land A \subseteq B \land finite(A) \land finite(B)  \Rightarrow card(A) \leq card(B)$

\item thm1 : $\forall f,g,h,k \cdot f \in \mathbb{Z} \pfun \mathbb{Z} \land g \in \mathbb{Z} \pfun \mathbb{Z} \land h \in dom(f) \land k \in dom(f) \land h \neq k \land g = f <+ \{h \mapsto f(k), k \mapsto f(h)\} \land finite(f) \land finite(g) \Rightarrow dom(g) = dom(f)$

\item thm6 : $\forall g,n,h,k \cdot n \in \mathbb{N} \land g \in 1..n \to \mathbb{Z} \land h \in dom(g) \land k \in dom(g) \Rightarrow \{h,k\} \dsub g = {i \cdot i \in 1..n \land i \neq h \land i \neq k | i \mapsto g(i)}$

\item thm2 : $\forall f,g,h,k,n \cdot n \in \mathbb{N} \land f \in 1..n \to \mathbb{Z} \land g \in 1..n \to \mathbb{Z} \land h \in dom(f) \land k \in dom(f) \land h \neq k \land g = f <+ \{h \mapsto f(k), k \mapsto f(h)\} \land finite(f) \land finite(g) \Rightarrow ran(g) = ran(f)$

\item thm3 : $\forall f,g,h,k,y,n \cdot n\in \mathbb{N} \land f \in 1..n \to \mathbb{Z} \land g \in 1..n \to \mathbb{Z} \land h \in dom(f) \land k \in dom(f) \land h \neq k \land g = f<+\{h \mapsto f(k), k \mapsto f(h) \} \land finite(f) \land finite(g) \land y \in ran(f) \Rightarrow card(\{u \cdot u \in dom(f) \land u \mapsto y \in f | u \}) =  card(\{x \cdot x \in dom(g) \land x \mapsto y \in g | x \})$

\item thm4 : $\forall f,g,h,k,n \cdot n \in \mathbb{N} \land f \in 1..n \to \mathbb{Z} \land g \in 1..n \to \mathbb{Z} \land h \in dom(f) \land k \in dom(f) \land h \neq k \land finite(f) \land finite(g) \land (\forall i \cdot i \in 1..n \land i \neq h \land i \neq k \Rightarrow g(i)=f(i)) \land g(h)=f(k) \land g(k)=f(h) \Rightarrow g = f<+\{h \mapsto f(k), k \mapsto f(h)\}$
\end{itemize}

\subsection{INITIALISATION/inv14/INV}
Demonstrate that : $finite(array)$ \\
By point out that : $\exists f \cdot f \in 1..size \bij array$ \\
I chose $f = \{i \cdot i\in 1..size | i \mapsto (i \mapsto array(i)) \}$, and then I repeat the removal of $\in$ in goal to simplify the PO.

\subsection{LOOP\_SWAP/inv14/INV}
Demonstrate that : $finite(a')$ \\
The process is the same as above, except that this time, I used the hypothese added to make the proving more easier : \\
$\{i \cdot i \in 1..size | i \mapsto a'(i)\} = a'$

\subsection{LOOP\_SWAP/inv13/INV}
Demonstrate that : \\
$\forall y \cdot i \in ran(array) \Rightarrow card(\{u \cdot u \in dom(array) \land u \mapsto y \in array | u\}) = card(\{x \cdot x \in dom(a') \land x \mapsto y \in a' | x\})$ \\

\paragraph{}
Use the hypothese : $card(\{u \cdot u \in dom(array) \land u \mapsto y \in array | u\}) = card(\{x \cdot x \in dom(a) \land x \mapsto y \in a | x\}) $. \\
We then need to prove : \\
$card(\{x \cdot x \in dom(a') \land x \mapsto y \in a' | x\}) = card(\{u \cdot u \in dom(a) \land u \mapsto y \in a | u\})$\\
Use the \textbf{thm4} to precise that $a' = a <+ \{indice \mapsto a(indice+1), indice+1 \mapsto a(indice) \}$ \\
Then use \textbf{thm3} to demonstrate $card(\{x \cdot x \in dom(a') \land x \mapsto y \in a' | x\}) = card(\{u \cdot u \in dom(a) \land u \mapsto y \in a | u\})$

Now the main point is, to prove all the added theorem.\textbf{thm1}, \textbf{thm5}, \textbf{thm7}, \textbf{thm9} are pretty much auto proved.

\subsection{thm8/THM}
This theorem is proved by using the original formular of $card(A \cup B \cup C)$, and the fact that intersection between A, B or C is empty.

\subsection{thm6/THM}
This theorem is proved by adding : \\
$\{h, k\} \dsub g = \{j \cdot j \in dom(g) \setminus \{h,k \} | j \mapsto g(j)  \} $

\subsection{thm2/THM}
This theorem was proved by re-define the overriding operartion, and then apply \textbf{thm6}, and then apply the classic process : \\
$A = B \Leftrightarrow A \subseteq B \land B \subseteq A $ \\
After re-define the overriding operation, we get the new PO : \\
$ran((\{h,k\} \dsub f \cup \{h \mapsto f(k), k \mapsto f(h) \})) = ran(f) $\\
This proposition $ran((\{h,k\} \dsub f \cup \{h \mapsto f(k), k \mapsto f(h) \})) \subseteq ran(f) $ is proved automatically. We need to prove : \\
$ran(f) \subseteq ran((\{h,k\} \dsub f \cup \{h \mapsto f(k), k \mapsto f(h) \}))$ \\
This is proved by point out : $\forall x0 \mapsto x \in f$, we prove that $x \in ran((\{h,k\} \dsub f \cup \{h \mapsto f(k), k \mapsto f(h) \}))$ \\
By using case distinction : $x0=h \lor x0=k \lor (x0 \neq h \land x0 \neq k)$, we can prove this easily.

\subsection{thm4/THM}
This theorem is proved by demonstrate 2 propositions :
\begin{itemize}
\item $g \subseteq f <+ \{h \mapsto f(k), k \mapsto f(h) \}$
\item $f <+ \{h \mapsto f(k), k \mapsto f(h) \} \subseteq g$
\end{itemize}

Then apply \textbf{thm5} on each side, and then divide after $dom(f)$ and $\{h,k\}$.

\subsection{thm3/THM}
This theorem is the main point, that lead to the demonstration of \textbf{LOOP\_SWAP/inv13/INV}. 

To do this, there're somethings we need to pay attention :\\
$\forall y \in rang(f) \cdot  \{u \cdot u \in 1..n \land  u \mapsto y \in f | u\} = \{u \cdot u \in 1..n \land u \neq h \land u \neq k \land  u \mapsto y \in f | u\} \cup \{u \cdot u=h \land u \mapsto y \in f | u \} \cup \{u \cdot u=k \land u \mapsto y \in f | u \} $ \\
We have 3 new subsets, and intersection between each pair, is null. Apply \textbf{thm8} for both $f$ and $g$ 's side. Then prove the followings : 
\begin{itemize}
\item $card( \{u \cdot u \in 1..n \land u \neq h \land u \neq k \land  u \mapsto y \in f | u\}) = card( \{u \cdot u \in 1..n \land u \neq h \land u \neq k \land  u \mapsto y \in g | u\})$

\item $card(\{u \cdot u=h \land u \mapsto y \in f | u \}) = card(\{u \cdot u=k \land u \mapsto y \in g | u \})$

\item $card(\{u \cdot u=k \land u \mapsto y \in f | u \}) = card(\{u \cdot u=h \land u \mapsto y \in g| u \})$
\end{itemize}

This can be done by using the definition of function $g$. 














