\section{Sort an array pairwised distinct values}

\begin{itemize}
\item Specification: SortArrayDistinctValueMachine
\item Implementation: SortArrayDistinctValueAscendingMachineImplementation
\end{itemize}

\paragraph{}
In this machine, the POs that need to be proved interactively is many, and the proving process is more complicated than before.

\subsection{LOOP\_SWAP/inv6/INV}
I added some hypotheses:
\begin{itemize}
\item $a'(last\_index+1) = a(last\_index+1)$
\item $last\_index +1 > indice+1$
\item $i=indice \lor i=indice+1 \lor (i \neq indice \land i \neq indice+1)$
\end{itemize} 

\subsection{LOOP\_SWAP/inv7/INV}
Hypothese added:\\
$i=indice \lor i=indice+1 \lor i < indice$\\
After that, I proved by case. First case, $i < indice$, add hypothese : $a'(i) = a(i)$. For two last cases, $i = indice \lor i=indice+1$, it was proved automatic.

\subsection{LOOP\_SWAP/inv10/INV}
Demonstrate:
$$
\forall i \cdot i \in last\_index+1..size \land i+1 \in last\_index+1..size \Rightarrow a'(i) \leq a'(i+1)
$$
\paragraph{}
I applied this hypothese for $i$ and $i+1$
$$
\forall i \cdot i \in 1..size \land \neg i=indice \land \neg i=indice+1\Rightarrow a'(i)=a(i)
$$
\paragraph{}
After that I applied this hypothese, and it was done.
$$
\forall i \cdot i \in last\_index+1..size \land i+1 \in last\_index+1..size \Rightarrow a(i) \leq a(i+1
$$

\subsection{LOOP\_SWAP/inv12/INV}
Demonstrate:
$$
\forall i \cdot i \in 1..size \Rightarrow (\exists j \cdot j \in 1..size \Rightarrow array(i)=a'(j))
$$

Applying case distinction : 
$$
i=indice \lor i=indice+1 \lor (i \neq indice \land i \neq indice+1)
$$
\paragraph{}
After that, each case was proved automatically.

\subsection{LOOP\_SWAP/act4/FIS}
Prove that :
\begin{align*}
&\exists a' \cdot a' \in 1..size \to array[1..size] \\ 
&\land (\forall i \cdot i \in 1..size \land i \neq indice \land i \neq indice+1 \Rightarrow a'(i)=a(i)) \\ 
&\land a'(indice) = a(indice+1) \land a'(indice+1) = a(indice)
\end{align*}

Here I chose that $ a' =  a <+ \{indice \mapsto a(indice+1), indice+1 \mapsto a(indice) \} $ \\
(I tried to define the function overriding symbol like in Rodin with $<+$) 

\paragraph{}
After that, the PO was proved automatically.

\subsection{BUBBLE\_SORT/inv10/INV}
Demonstrate:
$$
\forall i \cdot i \in last\_index..size \land i+1 \in last\_index..size \Rightarrow a(i) \leq a(i+1)
$$
\paragraph{}
I applied case distinction tactic, with $i = last\_index \lor i > last\_index$, and the PO was proved  automatically after that.

\subsection{RETURN/grd2/GRD}
Demonstrate:
\begin{align*}
&\forall i \cdot i \in dom(a) \land i+1 \in dom(a) \Rightarrow a(i) \leq a(i+1) \\
&\text{with hypothese: } \\
&(in\_loop=FALSE \land swap=FALSE \land indice=last\_index \land last\_index>1) \\
&\lor last\_index=1
\end{align*}

Apply 'proof by case'-tatic in the hypothese.

\subsubsection{First case: }
We have:
$$
(in\_loop=FALSE \land swap=FALSE \land indice=last\_index \land last\_index>1)
$$
Then do case distinction for i : $i < last\_index \lor i = last\_index \lor i > last\_index$
\paragraph{}
In case $i < last\_index$, apply hypothese : \\
$
\forall i \cdot i \in 1..last\_index \land i+1 \in 1..last\_index \Rightarrow a(i) \leq a(i+1)
$

\paragraph{}
In case $i = last\_index$, apply hypothese :\\
$
last\_index < size \Rightarrow (\forall i \cdot i \in 1..last\_index \Rightarrow a(i) \leq a(last\_index+1))
$

\paragraph{}
In case $i > last\_index$, apply hypothese :\\
$
\forall i \cdot i \in last\_index+1..size \land i \in last\_index+1..size \Rightarrow a(i) \leq a(i+1)
$

\subsubsection{Second case:}
We have: $last\_index = 1$ \\
\paragraph{}
Apply hypothese : \\
$
\forall i \cdot i \in last\_index+1..size \land i+1 \in last\_index+1..size \Rightarrow a(i) \leq a(i+1)
$

\paragraph{}
Then we do case distinction for i : $i=1 \lor i>1$

\subsection{RETURN/grd3/GRD}
Demonstrate : 
$$
\forall i \cdot i \in dom(array) \Rightarrow (\exists j \cdot j \in dom(a) \Rightarrow array(i) = a(j))
$$
This condition is enough to ensure that no member in \textbf{array} is missing in \textbf{a}, because the number of occurence of each member is 1. In the next part, sorting a random array, we have to ensure that the number of occurrences of each member in \textbf{array} and \textbf{a} are the same 

\subsection{liveness/THM}
Apply these hypothese (most of them are logic transformation) :
\begin{align*}
&(in\_loop=TRUE \land indice<last\_index \land a(indice)>a(indice+1) \land last\_index>1) \lor \\
&(in\_loop=TRUE \land indice<last\_index \land a(indice) \leq a(indice+1) \land last\_index>1)  \\
&\Leftrightarrow in\_loop=TRUE \land indice<last\_index \land last\_index>1
\end{align*}

\begin{align*}
&(in\_loop=TRUE \land indice<last\_index \land a(indice)>a(indice+1) \land last\_index>1) \lor \\
&(in\_loop=TRUE \land indice<last\_index \land a(indice) \leq a(indice+1) \land last\_index>1) \lor \\
&(in\_loop=TRUE \land indice=last\_index \land last\_index>1) \\
&\Leftrightarrow in\_loop=TRUE \land last\_index>1 \land indice \leq last\_index
\end{align*}

\begin{align*}
&(in\_loop=TRUE \land indice=last\_index \land last\_index>1) \lor \\
&(in\_loop=FALSE \land swap=FALSE \land indice=last\_index \land last\_index>1) \lor \\
&(in\_loop=FALSE \land indice=last\_index \land swap=TRUE \land last\_index>1) \\
&\Leftrightarrow indice=last\_index \land last\_index>1
\end{align*}









